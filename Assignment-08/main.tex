\let\negmedspace\undefined
\let\negthickspace\undefined
\documentclass[journal]{IEEEtran}
\usepackage[a5paper, margin=10mm, onecolumn]{geometry}
\usepackage{lmodern} % Ensure lmodern is loaded for pdflatex
\usepackage{tfrupee} % Include tfrupee package

\setlength{\headheight}{1cm} % Set the height of the header box
\setlength{\headsep}{0mm}     % Set the distance between the header box and the top of the text

\usepackage{gvv-book}
\usepackage{gvv}
\usepackage{cite}
\usepackage{amsmath,amssymb,amsfonts,amsthm}
\usepackage{algorithmic}
\usepackage{graphicx}
\usepackage{textcomp}
\usepackage{xcolor}
\usepackage{txfonts}
\usepackage{listings}
\usepackage{enumitem}
\usepackage{mathtools}
\usepackage{gensymb}
\usepackage{comment}
\usepackage[breaklinks=true]{hyperref}
\usepackage{tkz-euclide} 
\usepackage{listings}
\def\inputGnumericTable{}                                 
\usepackage[latin1]{inputenc}                                
\usepackage{color}                                            
\usepackage{array}                                            
\usepackage{longtable}                                       
\usepackage{calc}                                             
\usepackage{multirow}                                         
\usepackage{hhline}                                           
\usepackage{ifthen}                                           
\usepackage{lscape}

\begin{document}

\bibliographystyle{IEEEtran}
\vspace{3cm}

\title{11.16.3.21.2}
\author{EE24TECH11036 - Krishna Patil}
% \maketitle
% \newpage
% \bigskip
{\let\newpage\relax\maketitle}
\renewcommand{\thefigure}{\theenumi}
\renewcommand{\thetable}{\theenumi}
\setlength{\intextsep}{10pt} % Space between text and floats


\textbf{Question}: In a class of $60$ students, $30$ opted for NCC, $32$ opted for NSS and $24$ opted for both NCC and NSS. If one of these students is selected at random, find the probability that the student has opted neither NCC nor NSS. \\ \\
\solution
Define events $X$ and $Y$ as shown in the table \ref{table}, \\
\begin{table}[h!]    
  \centering
  \begin{tabular}[12pt]{ |c| c|} 
    \hline
    {Random Variable } & {Event}\\ 
    \hline
    $ X = 0 $ &  Student does not opt for NCC \\
    \hline 
    $ X = 1 $ & Student opts for NCC\\
    \hline
    $ Y = 0 $ & Student does not opt for NSS\\
    \hline   
    $ Y = 1 $ & Student opts for NSS \\
    \hline
\end{tabular}

  \caption{defining events}
  \label{table}
\end{table}
\newline Below are some posulates and theorems from boolean algebra : 
\begin{table}[h!]    
  \centering
  \begin{tabular}{|l|c l|c l|}
    \hline
    & (a) & & (b) & \\
    \hline
    Postulate 2 & $x + 0 = x$ & & $x \cdot 1 = x$ & \\
    \hline
    Postulate 5 & $x + x' = 1$ & & $x
    \cdot x' = 0$ & \\
    \hline
    Theorem 1 & $x + x = x$ & & $x \cdot x = x$ & \\
    \hline
    Theorem 2 & $x + 1 = 1$ & & $x \cdot 0 = 0$ & \\
    \hline
    Theorem 3, involution & $(x')' = x$ & & - & \\
    \hline
    Postulate 3, commutative & $x + y = y + x$ & & $xy = yx$ & \\
    \hline
    Theorem 4, associative & $x + (y + z) = (x + y) + z$ & & $x(yz) = (xy)z$ & \\
    \hline
    Postulate 4, distributive & $x(y + z) = xy + xz$ & & $x + yz = (x + y)(x + z)$ & \\
    \hline
    Theorem 5, DeMorgan & $(x + y)' = x' y'$ & & $(xy)' = x' + y'$ & \\
    \hline
    Theorem 6, absorption & $x + xy = x$ & & $x(x + y) = x$ & \\
    \hline
\end{tabular}

  \caption{Boolean Algebra}
  \label{table2}
\end{table}
\newline The axioms of probability are as follows:

\textbf{Non-Negativity Axiom:}
\[
P(A) \geq 0
\]
The probability of any event \( A \) is always non-negative.

\textbf{Normalization Axiom:}
\[
P(S) = 1
\]
The probability of the sample space \( S \) (i.e., the set of all possible outcomes) is 1.

\textbf{Additivity Axiom (Countable Additivity for Disjoint Events):}  
If \( A_1, A_2, A_3, \dots \) are mutually exclusive (disjoint) events, then:
\[
P(A_1 \cup A_2 \cup A_3 \cup \dots) = P(A_1) + P(A_2) + P(A_3) + \dots
\]

 For any two event A and B,
\begin{align}
	\because A + A^\prime &= 1 \\
	 AB + A^\prime B &= B \label{2} \\
	 \implies \pr{AB} + \pr{A^\prime B} &= \pr{B} \label{3} \\
	 \because B + B^\prime &= 1 \\
	 AB + AB^\prime &= A \label{5}\\
	 \implies \pr{AB} + \pr{AB^\prime} &= \pr{A} \label{6} \\
	 \text{adding } \eqref{2} \text{ and } \eqref{5} \\
	 A + B &= AB + AB + AB^\prime + A^\prime B  \\
	 A + B &= AB + AB^\prime + A^\prime B \\ 
	 \pr{A + B} &= \pr{AB} + \pr{AB^\prime} + \pr{A^\prime B} \label{10}\\
	 \text{Adding \eqref{3},\eqref{6} and \eqref{10} and cancelling same terms } \\
	 \pr{AB} &= \pr{A} + \pr{B} - \pr{A + B} \\
	 \because \pr{A^\prime B^\prime} &=  \pr{\brak{A + B}^\prime} \label{13} \\
	 \pr{A^\prime  B^\prime} &=  1 - \pr{A+B} \label{14}
\end{align}



From the given data in question,
    \begin{align}
        \pr{A} &= \frac{30}{60} \\
        \pr{B} &= \frac{32}{60} \\
        \pr{AB} &= \frac{24}{60}     
    \end{align}
Now using axioms of probability (boolean logic),
Thus, we write
    \begin{align}
	    \pr{A + B} &= \pr{A} +  \pr{B} - \pr{A B} \\
	                                 &= \frac{30}{60} + \frac{32}{60} - \frac{24}{60} \\
	                                 &= \frac{38}{60} \\
	     \pr{A^\prime  B^\prime} &=  1 - \pr{A+B} \brak{\because \eqref{13} \text{ and } \eqref{14}} \\ 
	                              &= 1- \frac{38}{60} = \frac{11}{30} 
    \end{align} 
So, the probablity $\pr{A^\prime  B^\prime}$ i.e., the probability that the student has opted neither NCC nor NSS is $\frac{11}{30} = 0.36667$.
Also after verifying using computational method to get the probability as 0.36680.
\end{document}


