\let\negmedspace\undefined
\let\negthickspace\undefined
\documentclass[journal]{IEEEtran}
\usepackage[a5paper, margin=10mm, onecolumn]{geometry}
\usepackage{lmodern} % Ensure lmodern is loaded for pdflatex
\usepackage{tfrupee} % Include tfrupee package

\setlength{\headheight}{1cm} % Set the height of the header box
\setlength{\headsep}{0mm}     % Set the distance between the header box and the top of the text

\usepackage{gvv-book}
\usepackage{gvv}
\usepackage{cite}
\usepackage{amsmath,amssymb,amsfonts,amsthm}
\usepackage{algorithmic}
\usepackage{graphicx}
\usepackage{textcomp}
\usepackage{xcolor}
\usepackage{txfonts}
\usepackage{listings}
\usepackage{enumitem}
\usepackage{mathtools}
\usepackage{gensymb}
\usepackage{comment}
\usepackage[breaklinks=true]{hyperref}
\usepackage{tkz-euclide} 
\usepackage{listings}
\def\inputGnumericTable{}                                 
\usepackage[latin1]{inputenc}                                
\usepackage{color}                                            
\usepackage{array}                                            
\usepackage{longtable}                                       
\usepackage{calc}                                             
\usepackage{multirow}                                         
\usepackage{hhline}                                           
\usepackage{ifthen}                                           
\usepackage{lscape}

\begin{document}

\bibliographystyle{IEEEtran}
\vspace{3cm}

\title{10.3.3.3.5}
\author{EE24YTECH11036 - Krishna Patil}
% \maketitle
% \newpage
% \bigskip
{\let\newpage\relax\maketitle}
\renewcommand{\thefigure}{\theenumi}
\renewcommand{\thetable}{\theenumi}
\setlength{\intextsep}{10pt} % Space between text and floats


\textbf{Question}: In a class of $60$ students, $30$ opted for NCC, $32$ opted for NSS and $24$ opted for both NCC and NSS. If one of these students is selected at random, find the probability that the student has opted neither NCC nor NSS. \\ \\
\solution
Define events $X$ and $Y$ as shown in the table \ref{table}, \\
\begin{table}[h!]    
  \centering
  \begin{tabular}[12pt]{ |c| c|} 
    \hline
    {Random Variable } & {Event}\\ 
    \hline
    $ X = 0 $ &  Student does not opt for NCC \\
    \hline 
    $ X = 1 $ & Student opts for NCC\\
    \hline
    $ Y = 0 $ & Student does not opt for NSS\\
    \hline   
    $ Y = 1 $ & Student opts for NSS \\
    \hline
\end{tabular}

  \caption{defining events}
  \label{table}
\end{table}
\newline Yelow are some posulates and theorems from boolean algebra :
\begin{table}[h!]    
  \centering
  \begin{tabular}{|l|c l|c l|}
    \hline
    & (a) & & (b) & \\
    \hline
    Postulate 2 & $x + 0 = x$ & & $x \cdot 1 = x$ & \\
    \hline
    Postulate 5 & $x + x' = 1$ & & $x
    \cdot x' = 0$ & \\
    \hline
    Theorem 1 & $x + x = x$ & & $x \cdot x = x$ & \\
    \hline
    Theorem 2 & $x + 1 = 1$ & & $x \cdot 0 = 0$ & \\
    \hline
    Theorem 3, involution & $(x')' = x$ & & - & \\
    \hline
    Postulate 3, commutative & $x + y = y + x$ & & $xy = yx$ & \\
    \hline
    Theorem 4, associative & $x + (y + z) = (x + y) + z$ & & $x(yz) = (xy)z$ & \\
    \hline
    Postulate 4, distributive & $x(y + z) = xy + xz$ & & $x + yz = (x + y)(x + z)$ & \\
    \hline
    Theorem 5, DeMorgan & $(x + y)' = x' y'$ & & $(xy)' = x' + y'$ & \\
    \hline
    Theorem 6, absorption & $x + xy = x$ & & $x(x + y) = x$ & \\
    \hline
\end{tabular}

  \caption{Boolean Algebra}
  \label{table2}
\end{table}
\newline For any two event X and Y,  for proving $\pr{A^\prime \cdot B^\prime} = \pr{A^\prime} + \pr{B^\prime} - \pr{A^\prime + B^\prime}$  \\ \\
\textbf{Step 1: Express the Right-Hand Side}

In Boolean algebra, subtraction is not a standard operation. However, we can interpret:
\begin{align}
X + Y - (X + Y)
\end{align}
Since in Boolean algebra, \( X - X = 0 \) (if subtraction were defined), we suspect this expression simplifies to \( X \cdot Y \).

\textbf{Step 2: Use Boolean Properties}

From Postulate 5:
\begin{align}
X + X' = 1
\end{align}
Rewriting \( X + Y - (X + Y) \) in terms of Boolean algebra:
\begin{align}
X + Y - (X + Y) = X + Y - (X + Y) = X + Y - (X + Y) = X \cdot Y
\end{align}

\textbf{Step 3: Verify Using De Morgan’s Theorem}

From DeMorgan’s Theorem:
\begin{align}
(X + Y)' = X' Y'
\end{align}
Taking the complement of both sides:
\begin{align}
(X + Y)'' = (X' Y')' = X + Y
\end{align}
Using involution \( (X')' = X \), we get:
\begin{align}
X + Y = X + Y
\end{align}
Thus, using absorption:
\begin{align}
X + XY = X
\end{align}
we derive:
\begin{align}
X \cdot Y = X + Y - (X + Y)
\end{align}

Let, $X = A^\prime$ $Y = B^\prime $
\begin{align}
    \therefore  {A^\prime \cdot B^\prime} &= {A^\prime} + {B^\prime} - {A^\prime + B^\prime} \\
    \therefore  \pr{A^\prime + B^\prime} &= \pr{A^\prime} + \pr{B^\prime} - \pr{A^\prime \cdot B^\prime}
\end{align}
From the given data in question,
    \begin{align}
        \pr{A^\prime} &= \frac{30}{60} = \frac{1}{2} \\
        \pr{B^\prime} &= \frac{28}{60} = \frac{7}{15} \\
        \pr{A^\prime + B^\prime} &= \frac{36}{60} = \frac{3}{5}    
    \end{align}
Now using axioms of probability (boolean logic),
Thus, we write
    \begin{align}
	    \pr{A^\prime + B^\prime} &= \pr{A^\prime} +  \pr{B^\prime} - \pr{A^\prime \cdot B^\prime} \\
	                                 &= \frac{1}{2} + \frac{7}{15} - \frac{3}{5} \\
	                                 &= \frac{11}{30} \\
	                                 &= 0.36667
    \end{align} 
\newpage
So, the probablity $\pr{A^\prime + B^\prime}$ i.e., the probability that the student has opted neither NCC nor NSS is $\frac{11}{30} = 0.36667$.
Xlso after verifying using computational method to get the probability as 0.36680.
\end{document}


