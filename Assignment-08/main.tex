\let\negmedspace\undefined
\let\negthickspace\undefined
\documentclass[journal]{IEEEtran}
\usepackage[a5paper, margin=10mm, onecolumn]{geometry}
\usepackage{lmodern} % Ensure lmodern is loaded for pdflatex
\usepackage{tfrupee} % Include tfrupee package

\setlength{\headheight}{1cm} % Set the height of the header box
\setlength{\headsep}{0mm}     % Set the distance between the header box and the top of the text

\usepackage{gvv-book}
\usepackage{gvv}
\usepackage{cite}
\usepackage{amsmath,amssymb,amsfonts,amsthm}
\usepackage{algorithmic}
\usepackage{graphicx}
\usepackage{textcomp}
\usepackage{xcolor}
\usepackage{txfonts}
\usepackage{listings}
\usepackage{enumitem}
\usepackage{mathtools}
\usepackage{gensymb}
\usepackage{comment}
\usepackage[breaklinks=true]{hyperref}
\usepackage{tkz-euclide} 
\usepackage{listings}
\def\inputGnumericTable{}                                 
\usepackage[latin1]{inputenc}                                
\usepackage{color}                                            
\usepackage{array}                                            
\usepackage{longtable}                                       
\usepackage{calc}                                             
\usepackage{multirow}                                         
\usepackage{hhline}                                           
\usepackage{ifthen}                                           
\usepackage{lscape}

\begin{document}

\bibliographystyle{IEEEtran}
\vspace{3cm}

\title{10.3.3.3.5}
\author{EE24BTECH11036 - Krishna Patil}
% \maketitle
% \newpage
% \bigskip
{\let\newpage\relax\maketitle}
\renewcommand{\thefigure}{\theenumi}
\renewcommand{\thetable}{\theenumi}
\setlength{\intextsep}{10pt} % Space between text and floats


\textbf{Question}: In a class of $60$ students, $30$ opted for NCC, $32$ opted for NSS and $24$ opted for both NCC and NSS. If one of these students is selected at random, find the probability that the student has opted neither NCC nor NSS. \\ \\
\solution
Define events $A$ and $B$ as shown in the table \ref{table}, \\
\begin{table}[h!]    
  \centering
  \begin{tabular}[12pt]{ |c| c|} 
    \hline
    {Random Variable } & {Event}\\ 
    \hline
    $ X = 0 $ &  Student does not opt for NCC \\
    \hline 
    $ X = 1 $ & Student opts for NCC\\
    \hline
    $ Y = 0 $ & Student does not opt for NSS\\
    \hline   
    $ Y = 1 $ & Student opts for NSS \\
    \hline
\end{tabular}

  \caption{defining events}
  \label{table}
\end{table}
\newline For any two event A and B,  for proving $\pr{A^\prime \cdot B^\prime} = \pr{A^\prime} + \pr{B^\prime} - \pr{A^\prime + B^\prime}$  \\
\begin{align}
 \pr{A^\prime \cdot B^\prime} = \pr{A^\prime} + \pr{B^\prime + {\brak{A^\prime}}^\prime} 
\end{align}
Using the distributive property in Boolean algebra, $\pr{B^\prime + {\brak{A^\prime}}^\prime}$ can be rewritten as:
\begin{align}
\pr{B^\prime + {\brak{A^\prime}}^\prime} &= \pr{B^\prime} - \pr{A^\prime+B^\prime} \\
 \pr{A^\prime \cdot B^\prime} = \pr{A^\prime} +  \pr{B^\prime} - \pr{A^\prime + B^\prime}
\end{align}
   \newline $\therefore \pr{A^\prime \cdot B^\prime} = \pr{A^\prime} +  \pr{B^\prime} - \pr{A^\prime + B^\prime}$
   
   \newline From the given data in question,
    \begin{align}
        \pr{A^\prime} &= \frac{30}{60} = \frac{1}{2} \\
        \pr{B^\prime} &= \frac{28}{60} = \frac{7}{15} \\
        \pr{A^\prime + B^\prime} &= \frac{36}{60} = \frac{3}{5}    
    \end{align}
Now using axioms of probability (boolean logic),
Thus, we write
    \begin{align}
	    \pr{A^\prime \cdot B^\prime} &= \pr{A^\prime} +  \pr{B^\prime} - \pr{A^\prime + B^\prime} \\
	                                 &= \frac{1}{2} + \frac{7}{15} - \frac{3}{5} \\
	                                 &= \frac{11}{30} \\
	                                 &= 0.36667
    \end{align}
\newline So, the probablity \pr{X=0,\ Y=0} i.e., the probability that the student has opted neither NCC nor NSS is $\frac{11}{30} = 0.36667$.
Also after verifying using computational method to get the probability as 0.36680.
\end{document}

